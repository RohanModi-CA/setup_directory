\usepackage[bottom]{footmisc}
\usepackage{pdfpages}
\usepackage[framemethod=TikZ]{mdframed}
\usepackage{booktabs}
\usepackage{amsmath}
\usepackage{amssymb}
\usepackage{physics}
\usepackage{braket}
\usepackage{cancel}
\usepackage[shortlabels]{enumitem}
\usepackage{fancyhdr}
\usepackage{tcolorbox}
\usepackage{ifthen}

% \everymath{\displaystyle}

\renewcommand{\l}{\lambda}
\newcommand{\p}{\phi}
\newcommand{\w}{\omega}
\renewcommand{\r}{\rho}
\newcommand{\B}{\beta}
\newcommand{\e}{\varepsilon}
\newcommand{\g}{\gamma}
\newcommand{\C}{\zeta}
\newcommand{\Del}{\nabla}  
\renewcommand{\a}{\alpha}
\renewcommand{\d}{\delta}
\newcommand{\Let}{\text{Let } }  
\newcommand{\R}{\mathbb{R}}
\renewcommand{\&}{&\text{ } }  
\renewcommand{\and}{\text{ and } }  
\pagenumbering{gobble}
\setlength{\parindent}{0pt}
\counterwithin*{equation}{section}
\newcommand{\hr}{\bigskip\hrule\bigskip}
\newcommand{\midskip}{\medskip}

\usepackage{newunicodechar}
\newunicodechar{ℯ}{{e}} % No \makebox for better minted handling


\vfuzz=100pt
\let\oldexists\exists \renewcommand{\exists}{\text{ }\oldexists\text{ }}
\let\oldsection\section \renewcommand{\section}{\setcounter{equation}{0}\oldsection}
\let\oldsubsection\subsection \renewcommand{\subsection}{\setcounter{equation}{0}\oldsubsection}


\definecolor{lightergrey}{HTML}{E8E8E8}
\usepackage{fontspec}
\usepackage[left=0.8in, right=0.8in]{geometry}

\setmonofont{DejaVu Sans Mono}



\definecolor{lightergrey}{HTML}{E8E8E8}

\setminted{
  style=default,
  bgcolor=lightgray,         % Light gray background
  fontsize=\scriptsize,       % Smaller font size
  breaklines=true,           % Wrap long lines
  breakanywhere=true,
  frame=lines,                % Frame around the code block
  linenos=true,              % Show line numbers
  numbersep=5pt,             % Space between line numbers and code
  tabsize=2,                 % Tab size
  showspaces=false,          % Don't show spaces
  showtabs=false,            % Don't show tabs
  gobble=0,                  % Remove indentation 
  mathescape=true,           % Allow inline math mode
  python3=true,             % Python 3 syntax highlighting (adjust as needed)
  % ... other options can be added here as needed ...
}

\setminted[inputminted]{
  style=default,
  bgcolor=red,         % Light gray background
  fontsize=\scriptsize,       % Smaller font size
  breaklines=true,           % Wrap long lines
  breakanywhere=true,
  frame=lines,                % Frame around the code block
  linenos=true,              % Show line numbers
  numbersep=5pt,             % Space between line numbers and code
  tabsize=2,                 % Tab size
  showspaces=false,          % Don't show spaces
  showtabs=false,            % Don't show tabs
  gobble=0,                  % Remove indentation 
  mathescape=true,           % Allow inline math mode
  python3=true,             % Python 3 syntax highlighting (adjust as needed)
  % ... other options can be added here as needed ...
}

\setmintedinline{
  bgcolor=lightergrey,
  fontsize=\footnotesize,
  breaklines=true,
  breaksymbolleft=\tiny$\hookrightarrow$
}

\newmintinline[pythons]{python}{python3=true}
\newmintinline{bash}{}

\usemintedstyle{xcode}
\setminted[python]{style=xcode}



% Define a custom environment with an optional argument for remarks
\newtcolorbox[auto counter, number within=section]{remarkbox}[1][]{
    colback=gray!10, % Light grey background
    colframe=gray!50, % Slightly darker grey border
    coltitle=black,
    boxrule=0.5pt, % Border thickness
    left=1em, % Left padding
    right=1em, % Right padding
    top=1em, % Top padding
    bottom=1em, % Bottom padding
    title=Remark \thetcbcounter\ifstrempty{#1}{}{ -- #1}, % Header with optional text
    fonttitle=\bfseries, % Bold header text
}

% Define a custom environment with an optional argument for remarks
\newtcolorbox[auto counter, number within=section]{resultbox}[1][]{
    colback=gray!10, % Light grey background
    colframe=gray!50, % Slightly darker grey border
    coltitle=black,
    boxrule=0.5pt, % Border thickness
    left=1em, % Left padding
    right=1em, % Right padding
    top=1em, % Top padding
    bottom=1em, % Bottom padding
    title=Result \thetcbcounter\ifstrempty{#1}{}{ -- #1}, % Header with optional text
    fonttitle=\bfseries, % Bold header text
}

\newcounter{theo}[section]
\setcounter{theo}{0}
\renewcommand{\thetheo}{\arabic{section}.\arabic{theo}}

\newenvironment{theo}[1][]{%
  \refstepcounter{theo}%
  \ifstrempty{#1}%
  {\mdfsetup{%
    frametitle={%
      \tikz[baseline=(current bounding box.east),outer sep=0pt]
      \node[anchor=east,rectangle,fill=blue!20, text width=0.9\linewidth]
      {\strut \parbox{\linewidth}{Theorem~\thetheo}};}}
  }%
  {\mdfsetup{%
    frametitle={%
      \tikz[baseline=(current bounding box.east),outer sep=0pt]
      \node[anchor=east,rectangle,fill=blue!20, text width=0.9\linewidth]
      {\strut \parbox{\linewidth}{Theorem~\thetheo:~#1}};}}%
  }%
  \mdfsetup{innertopmargin=10pt,linecolor=blue!20,%
    linewidth=2pt,topline=true,%
    frametitleaboveskip=\dimexpr-\ht\strutbox\relax%  No additional skip needed with parbox approach
  }
  \begin{mdframed}[]\relax%
  }
  {\end{mdframed}}


% now the question environment:
\newcounter{question}[section]
\setcounter{question}{0}
\renewcommand{\thequestion}{}

\newenvironment{question}[1][]{%
  \refstepcounter{question}%
  \ifstrempty{#1}%
  {\mdfsetup{%
    frametitle={%
      \tikz[baseline=(current bounding box.east),outer sep=0pt]
      \node[anchor=east,rectangle,fill=black!95, text width=0.9\linewidth]
      {\strut \parbox{\linewidth}{Question~\thequestion}};}}
  }%
  {\mdfsetup{%
    frametitle={%
      \tikz[baseline=(current bounding box.east),outer sep=0pt]
      \node[anchor=east,rectangle,fill=black!95, text width=0.9\linewidth]
{\strut \parbox{\linewidth}{\color{white} Question\thequestion~#1}};}}%
  }%
  \mdfsetup{innertopmargin=10pt,linecolor=black!95,%
    linewidth=2pt,topline=true,%
    frametitleaboveskip=\dimexpr-\ht\strutbox\relax%  No additional skip needed with parbox approach
  }
  \begin{mdframed}[]\relax%
  }
  {\end{mdframed}}


\usepackage{tcolorbox}
\newtcolorbox{myheadingbox}{colback=black!0!white, colframe=black, width=\textwidth, boxrule=0.75mm, arc=0mm, boxsep=5pt, left=0pt}

\newenvironment{myheader}[1]
{
    \begin{center}
    \begin{myheadingbox}
    \textbf{\large #1}
}
{
    \end{myheadingbox}
    \end{center}
}

\usepackage{xcolor}
\usepackage{mdframed}

% Custom environment for definitions
\newmdenv[
    backgroundcolor=red!10,
    linecolor=red!50,
    linewidth=1pt,
    roundcorner=6pt,
    innerleftmargin=10pt,
    innerrightmargin=10pt,
    innertopmargin=3pt,
    innerbottommargin=3pt,
    skipabove=1pt,
    skipbelow=1pt
]{definitionbox}

\newenvironment{definition}[1][]{%
    \begin{definitionbox}
    \textbf{Definition#1 -} %
}{%
    \end{definitionbox}
}
